The most fundamental question with respect to regular expressions is asking if a given input word is accepted by a given regular expression. This is called the membership problem, and it returns a simple boolean answer. Most practical solutions to this problem have significantly deviated from their theoretical roots, by extending regular expressions to accept non-regular languages, as well as using exponential backtracking approaches rather than current developments in the theoretical field. As the theory has evolved, publications have been accompanied by asymptotic analyses. But to our knowledge, a real-world practical study of these improvements has not been conducted.

This research will test various algorithms for solving the membership problem on practical regular expressions used by developers. There are two distinct classes of algorithms: backtracking and partial derivatives can be computed directly on a regular expression, and automata solutions built using Thompson, Glushkov, position, follow, and partial derivative constructions. Which algorithms applied to the membership problem are the fastest with respect to practical regular expressions? How does the exponential backtracking algorithm compare with its theoretical counterparts? If you were writing a modern regular expression library today, which algorithms should be included for membership evaluation?

We begin in Chapter \ref{ch:Background} with a basic overview of essential regular expression related topics. Chapter \ref{ch:Literature Review} goes beyond widely understood topics and explores techniques and constructions useful for the rest of the project. In Chapter \ref{ch:Programming Tools} useful libraries and programming tools are briefly explained. The methods and implementation details are given in Chapter \ref{ch:Methodology}. Finally, in Chapters \ref{ch:Results} and \ref{ch:Conclusions} we discuss the results of the experiments and provide some directions for further research.